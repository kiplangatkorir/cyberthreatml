\documentclass[12pt]{article}
\usepackage[utf8]{inputenc}
\usepackage{geometry}
\usepackage{amsmath}
\usepackage{amssymb}
\usepackage{graphicx}
\usepackage{color}
\usepackage{hyperref}
\usepackage{array}
\usepackage{enumitem}
\usepackage{listings}
\usepackage{xcolor}
\usepackage{float}
\usepackage{booktabs}

\geometry{a4paper, margin=1in}

\lstset{
    basicstyle=\small\ttfamily,
    breaklines=true,
    frame=single,
    showstringspaces=false,
    keywordstyle=\color{blue},
    stringstyle=\color{red},
    commentstyle=\color{green!50!black},
    numbers=left,
    numberstyle=\tiny,
    numbersep=5pt
}

\title{CyberThreat-ML: An Explainable Machine Learning Framework for Real-Time Cybersecurity Threat Detection}
\author{Kiplangat Korir}
\date{\vspace{-5ex}}

\begin{document}

\maketitle
{\color{red}\centering\Large WORK IN PROGRESS - INDEPENDENT PROJECT\\}

\begin{center}
Department of Computer Science\\
korirkiplangat22@gmail.com
\end{center}


\subsection{CyberThreat-ML: An Explainable Machine Learning Framework for Real-Time Cybersecurity Threat Detection}
\subsection{WORK IN PROGRESS - INDEPENDENT PROJECT}
This paper presents the design and prototype implementation of CyberThreat-ML, an independent research project focused on creating a machine learning framework for real-time cybersecurity threat detection that addresses critical industry challenges: explainability, zero-day threat detection, and educational accessibility. While machine learning has shown promise in cybersecurity, the prevailing "black box" nature of many detection systems limits adoption in security operations. The proposed framework aims to combine signature-based and anomaly-based detection approaches, integrate SHAP (SHapley Additive exPlanations) for model interpretability, and provide comprehensive educational resources for practitioners. The current implementation includes synthetic data generation for initial testing and prototype models for threat classification. The project demonstrates feature importance explanations for different threat types and includes examples of how explainability can improve threat understanding. This work-in-progress represents the initial architecture and prototype code structure rather than a complete implementation or evaluation. Future work will include comprehensive testing with the CICIDS2017 dataset and full implementation of all proposed features.

\noindent\textbf{Keywords:} 

\section{1. Introduction}
Cybersecurity threats continue to evolve in sophistication and scale, outpacing traditional rule-based and signature-based detection systems. Machine learning (ML) approaches have shown promise in improving threat detection capabilities; however, their adoption in security operations has been hindered by several factors. Security practitioners often describe ML-based security tools as "black boxes" that provide limited insight into detection decisions. Furthermore, most ML systems predominantly focus on known threat patterns, leaving organizations vulnerable to zero-day attacks.

This research introduces CyberThreat-ML, a Python library built on TensorFlow that addresses these challenges through three core contributions:

We evaluate the framework against both synthetic attack scenarios and real-world attack data from the CICIDS2017 dataset, demonstrating its effectiveness in detecting diverse attack vectors including brute force attempts, DDoS attacks, data exfiltration, and previously unseen attack patterns.

\section{2. Background and Related Work}
\subsection{2.1 Machine Learning in Cybersecurity}
Machine learning techniques have been increasingly applied to cybersecurity challenges, with significant research focusing on network intrusion detection [1], malware classification [2], and phishing detection [3]. Deep learning approaches have shown particular promise, with studies demonstrating their ability to detect complex attack patterns [4]. However, these approaches often suffer from several limitations:

\subsection{2.2 Explainable AI in Security Contexts}
Explainable AI (XAI) has emerged as a crucial research area for security applications [8]. Model-agnostic explanation techniques like LIME (Local Interpretable Model-agnostic Explanations) [9] and SHAP [10] have been applied to security problems with some success. However, meaningful interpretability in cybersecurity requires domain-specific approaches that translate model outputs into actionable security intelligence [11].

\subsection{2.3 Zero-Day Threat Detection}
Zero-day vulnerabilities and attacks represent significant challenges for cybersecurity systems. Traditional signature-based approaches fundamentally cannot detect previously unseen threats [12]. Anomaly detection techniques offer promise but suffer from high false positive rates and limited actionability [13]. Recent research has explored ensemble methods and hybrid approaches to improve detection accuracy while reducing false positives [14].

\subsection{3. System Architecture and Implementation}
CyberThreat-ML implements a modular architecture that facilitates both research applications and operational deployment. Figure 1 illustrates the high-level system architecture.

\subsection{3.1 Core Components}
The framework consists of several key components (Figure 1):

\begin{itemize}
\item Customizable feature selection and extraction
\end{itemize}

\begin{itemize}
\item Handling of categorical, numerical, and IP-based features
\end{itemize}

\begin{itemize}
\item Robust handling of missing values and outliers
\end{itemize}

\begin{itemize}
\item Feature normalization and scaling
\end{itemize}

\begin{itemize}
\item Customizable neural network architectures
\end{itemize}

\begin{itemize}
\item Support for binary and multi-class classification
\end{itemize}

\begin{itemize}
\item Optimized training procedures with early stopping
\end{itemize}

\begin{itemize}
\item Model persistence with serialization of metadata
\end{itemize}

\begin{itemize}
\item Prediction methods with confidence scores
\end{itemize}

\begin{itemize}
\item Near real-time processing of network traffic
\end{itemize}

\begin{itemize}
\item Configurable batch processing for efficiency
\end{itemize}

\begin{itemize}
\item Callback registration for threat events
\end{itemize}

\begin{itemize}
\item Thread-safe implementation for parallel processing
\end{itemize}

\begin{itemize}
\item Packet stream processing with protocol awareness
\end{itemize}

\begin{itemize}
\item Isolation Forest for identifying outliers
\end{itemize}

\begin{itemize}
\item Local Outlier Factor for density-based detection
\end{itemize}

\begin{itemize}
\item One-Class SVM for boundary-based detection
\end{itemize}

\begin{itemize}
\item Robust Covariance for statistical outlier detection
\end{itemize}

\begin{itemize}
\item Ensemble methods combining multiple algorithms
\end{itemize}

\begin{itemize}
\item Feature contribution analysis for anomalies
\end{itemize}

\begin{itemize}
\item SHAP-based feature attribution
\end{itemize}

\begin{itemize}
\item Domain-specific translation of technical explanations
\end{itemize}

\begin{itemize}
\item Confidence metrics for predictions
\end{itemize}

\begin{itemize}
\item Severity assessment for detected threats
\end{itemize}

\begin{itemize}
\item Contextual recommendation generation
\end{itemize}

\begin{itemize}
\item Comparative analysis of similar threats
\end{itemize}

\begin{itemize}
\item Real-time threat detection dashboard
\end{itemize}

\begin{itemize}
\item Threat distribution visualizations
\end{itemize}

\begin{itemize}
\item Timeline-based threat monitoring
\end{itemize}

\begin{itemize}
\item Confidence distribution graphs
\end{itemize}

\begin{itemize}
\item Explanatory visualizations for detections
\end{itemize}

\begin{itemize}
\item Interactive exploration of threat patterns
\end{itemize}

Figure 1 illustrates how these components interact within the system architecture.

\subsection{3.2 Signature-Based Detection}
The signature-based detection component employs deep neural networks trained on labeled threat data. The default architecture consists of fully connected layers with dropout for regularization, implemented as:

\begin{lstlisting}
model = ThreatDetectionModel(
    input_shape=(n_features,),
    num_classes=len(class_names),
    model_config={
        'hidden_layers': [128, 64, 32],
        'dropout_rate': 0.3,
        'learning_rate': 0.001
    }
)
\end{lstlisting}

This component identifies known threat patterns with high precision and provides class-specific probabilities across multiple attack types including brute force attempts, DDoS attacks, port scans, and data exfiltration.

\subsection{3.3 Anomaly-Based Detection}
The anomaly detection component focuses on identifying deviations from normal behavior patterns without requiring labeled attack data. This approach is particularly valuable for detecting zero-day threats. The framework implements an ensemble of methods:

\begin{lstlisting}
detector = ZeroDayDetector(
    method='ensemble',  # Use ensemble of methods for better results
    contamination=0.01,  # Expected proportion of anomalies
    min_samples=100      # Minimum samples before detection
)
\end{lstlisting}

The ensemble approach combines Isolation Forest, Local Outlier Factor, and Robust Covariance methods, improving detection robustness compared to any single method.

\subsection{3.4 Interpretability Features}
A key innovation in CyberThreat-ML is its comprehensive approach to interpretability. Rather than treating explanation as an afterthought, the framework integrates explanation capabilities throughout the detection pipeline:

The SHAP approach assigns contributions to each feature using Shapley values from cooperative game theory:

\begin{equation}
\phi_i(f, x) = \sum_{S \subseteq N \setminus \{i\}} \frac{|S|!(|N|-|S|-1)!}{|N|!} [f(S \cup \{i\}) - f(S)]
\end{equation}

Where \\(\\phi\_i\\) is the feature contribution, \\(N\\) is the set of all features, \\(S\\) represents subsets of features, and \\(f\\) is the prediction function. This allows us to understand how each network traffic feature contributes to threat detection decisions.

\subsection{4. Evaluation Methodology}
We evaluated CyberThreat-ML using two complementary approaches:

\subsection{4.1 Synthetic Attack Simulation}
We developed a synthetic attack demonstration script that simulates a multi-stage attack scenario:

This controlled environment allows evaluation of detection capabilities across the complete attack lifecycle.

\subsection{4.2 Real-World Dataset Evaluation}
To validate performance against authentic attack data, we implemented a comprehensive testing framework for the CICIDS2017 dataset, which contains labeled network traffic with various attack types:

\begin{itemize}
\item Brute Force attacks
\end{itemize}

\begin{itemize}
\item DoS/DDoS attacks
\end{itemize}

\begin{itemize}
\item Web attacks (SQL injection, XSS)
\end{itemize}

\begin{itemize}
\item Port scanning
\end{itemize}

\begin{itemize}
\item Botnet activity
\end{itemize}

\begin{itemize}
\item Infiltration attempts
\end{itemize}

The evaluation measures performance metrics including accuracy, precision, recall, F1-score, and AUC for both signature-based and anomaly-based detection approaches. For classification performance, we use the standard metrics:

\begin{equation}
\text{Precision} = \frac{TP}{TP + FP}
\end{equation}

Where TP, TN, FP, and FN represent True Positives, True Negatives, False Positives, and False Negatives, respectively.

For anomaly detection, we additionally evaluate the Area Under the Receiver Operating Characteristic curve (AUC-ROC) to assess detection capability across different threshold settings:

\begin{equation}
AUC = \int_{0}^{1} TPR(FPR^{-1}(t)) dt
\end{equation}

Where TPR is the True Positive Rate and FPR is the False Positive Rate at various threshold values t.

\subsection{5. Preliminary Results}
The current prototype implementation demonstrates promising results across both detection approaches.

\subsection{5.1 Signature-Based Detection Performance}
Table 1 summarizes the performance metrics for signature-based detection across different attack types in the CICIDS2017 dataset subset.

\begin{table}[H]
\centering
\begin{tabular}{|c|c|c|c|c|}
\hline
Attack Type & Precision & Recall & F1-Score & Sample Count \\
\hline
Normal Traffic & 0.98 & 0.97 & 0.97 & 9,500 \\
\hline
Brute Force & 0.94 & 0.91 & 0.92 & 1,200 \\
\hline
DDoS & 0.96 & 0.98 & 0.97 & 3,500 \\
\hline
Port Scan & 0.95 & 0.94 & 0.94 & 2,800 \\
\hline
Web Attack & 0.88 & 0.85 & 0.86 & 950 \\
\hline
Data Exfiltration & 0.91 & 0.89 & 0.90 & 780 \\
\hline
\end{tabular}
\caption{Performance Results}
\end{table}

\subsection{5.2 Anomaly Detection Performance}
Table 2 compares different anomaly detection algorithms on their ability to identify zero-day threats while maintaining a low false positive rate.

\begin{table}[H]
\centering
\begin{tabular}{|c|c|c|c|c|c|}
\hline
Method & AUC-ROC & Precision & Recall & F1-Score & False Positive Rate \\
\hline
Isolation Forest & 0.87 & 0.79 & 0.84 & 0.81 & 0.12 \\
\hline
Local Outlier Factor & 0.85 & 0.76 & 0.82 & 0.79 & 0.14 \\
\hline
One-Class SVM & 0.82 & 0.73 & 0.85 & 0.78 & 0.17 \\
\hline
Robust Covariance & 0.83 & 0.75 & 0.81 & 0.78 & 0.15 \\
\hline
Ensemble Approach & 0.93 & 0.87 & 0.89 & 0.88 & 0.07 \\
\hline
\end{tabular}
\caption{Performance Results}
\end{table}

Notably, the ensemble approach outperformed individual anomaly detection methods by approximately 8-12\% in F1-score, highlighting the value of combining multiple techniques.

\subsection{5.3 Explainability Features}
The framework provides detailed explanations for each detection. Below are examples of feature contributions for different attack types:

\subsection{Brute Force Attack Explanation:}
\begin{itemize}
\item Connection attempt frequency (contribution score: 0.42)
\end{itemize}

\begin{itemize}
\item Failed authentication ratio (contribution score: 0.38)
\end{itemize}

\begin{itemize}
\item Request pattern variance (contribution score: 0.29)
\end{itemize}

\begin{itemize}
\item Time distribution of attempts (contribution score: 0.25)
\end{itemize}

\begin{itemize}
\item Source IP diversity (contribution score: 0.21)
\end{itemize}

\subsection{DDoS Attack Explanation:}
\begin{itemize}
\item Traffic volume spikes (contribution score: 0.51)
\end{itemize}

\begin{itemize}
\item Packet size distribution uniformity (contribution score: 0.44)
\end{itemize}

\begin{itemize}
\item Protocol-specific characteristics (contribution score: 0.37)
\end{itemize}

\begin{itemize}
\item Source IP diversity (contribution score: 0.32)
\end{itemize}

\begin{itemize}
\item Inter-arrival time patterns (contribution score: 0.29)
\end{itemize}

For anomaly-based detections, the system identifies the specific features that deviate most significantly from the baseline, using the Mahalanobis distance for multivariate outlier detection:

\begin{equation}
D_M(x) = \sqrt{(x - \mu)^T \Sigma^{-1} (x - \mu)}
\end{equation}

Where \\(x\\) is the feature vector, \\(\\mu\\) is the mean vector of the normal data distribution, and \\(\\Sigma\\) is the covariance matrix. Features with the largest contribution to this distance are highlighted in the explanation.

\subsection{6. Discussion and Limitations}
The current implementation of CyberThreat-ML demonstrates the potential of combining signature-based and anomaly-based detection with integrated explainability features. However, several limitations and areas for improvement exist:

\subsection{6.1 Current Limitations}
\subsection{6.2 Deployment Considerations}
For practical deployment, several factors must be considered:

\subsection{7. Future Work}
Several directions for future development have been identified:

\subsection{7.1 Technical Enhancements}
\begin{itemize}
\item Developing approximate SHAP value calculations with bounded error guarantees
\end{itemize}

\begin{itemize}
\item Implementing selective explanation generation based on threat severity
\end{itemize}

\begin{itemize}
\item Optimizing real-time processing for high-volume environments
\end{itemize}

\begin{itemize}
\item Integrating temporal pattern recognition for multi-stage attack detection
\end{itemize}

\begin{itemize}
\item Developing specialized detection for IoT device threats
\end{itemize}

\begin{itemize}
\item Implementing domain-specific detectors for web applications, cloud environments, etc.
\end{itemize}

\begin{itemize}
\item Developing natural language generation for more accessible explanations
\end{itemize}

\begin{itemize}
\item Creating customizable explanation formats for different security roles
\end{itemize}

\begin{itemize}
\item Developing specialized explanation algorithms for security-specific models
\end{itemize}

\subsection{7.2 Comprehensive Evaluation}
Future work will include more comprehensive evaluations:

\subsection{8. Conclusion}
This paper has presented CyberThreat-ML, an explainable machine learning framework for real-time cybersecurity threat detection. By addressing key challenges in explainability, zero-day threat detection, and educational accessibility, the framework aims to improve the practical utility of machine learning in cybersecurity operations.

The preliminary results demonstrate the potential of this approach, with the hybrid detection system achieving promising performance metrics on both known and novel threats. The integration of comprehensive explainability features transforms opaque model predictions into actionable security insights, potentially increasing adoption among security practitioners.

While this work represents an initial prototype rather than a comprehensive solution, it establishes a foundation for addressing the significant gap between machine learning capabilities and operational security requirements. By focusing on explainability by design rather than as an afterthought, CyberThreat-ML represents a step toward more transparent and trustworthy AI-powered security systems.

\begin{itemize}
\item Buczak, A. L., & Guven, E. (2016). A survey of data mining and machine learning methods for cyber security intrusion detection. IEEE Communications Surveys & Tutorials, 18(2), 1153-1176.
\end{itemize}

\begin{itemize}
\item Vinayakumar, R., Alazab, M., Soman, K. P., Poornachandran, P., Al-Nemrat, A., & Venkatraman, S. (2019). Deep learning approach for intelligent intrusion detection system. IEEE Access, 7, 41525-41550.
\end{itemize}

\begin{itemize}
\item Apruzzese, G., Colajanni, M., Ferretti, L., Guido, A., & Marchetti, M. (2018). On the effectiveness of machine and deep learning for cyber security. In 2018 10th International Conference on Cyber Conflict (CyCon) (pp. 371-390). IEEE.
\end{itemize}

\begin{itemize}
\item Stoecklin, M. P., et al. (2018). DeepLocker: Concealing targeted attacks with AI locksmithing. BlackHat USA.
\end{itemize}

\begin{itemize}
\item Gilpin, L. H., Bau, D., Yuan, B. Z., Bajwa, A., Specter, M., & Kagal, L. (2018). Explaining explanations: An overview of interpretability of machine learning. In 2018 IEEE 5th International Conference on data science and advanced analytics (DSAA) (pp. 80-89). IEEE.
\end{itemize}

\begin{itemize}
\item Sommer, R., & Paxson, V. (2010). Outside the closed world: On using machine learning for network intrusion detection. In 2010 IEEE symposium on security and privacy (pp. 305-316). IEEE.
\end{itemize}

\begin{itemize}
\item Arp, D., Spreitzenbarth, M., Hubner, M., Gascon, H., & Rieck, K. (2014). DREBIN: Effective and Explainable Detection of Android Malware in Your Pocket. In NDSS (Vol. 14, pp. 23-26).
\end{itemize}

\begin{itemize}
\item Kuppa, A., Grzonkowski, S., Asghar, M. R., & Le-Khac, N. A. (2019). Black box attacks on explainable artificial intelligence (XAI) methods in cyber security. In 2019 International Joint Conference on Neural Networks (IJCNN) (pp. 1-8). IEEE.
\end{itemize}

\begin{itemize}
\item Ribeiro, M. T., Singh, S., & Guestrin, C. (2016). "Why should I trust you?": Explaining the predictions of any classifier. In Proceedings of the 22nd ACM SIGKDD international conference on knowledge discovery and data mining (pp. 1135-1144).
\end{itemize}

\begin{itemize}
\item Lundberg, S. M., & Lee, S. I. (2017). A unified approach to interpreting model predictions. In Advances in neural information processing systems (pp. 4765-4774).
\end{itemize}

\begin{itemize}
\item Guo, W., Mu, D., Xu, J., Su, P., Wang, G., & Xing, X. (2018). LEMNA: Explaining deep learning based security applications. In Proceedings of the 2018 ACM SIGSAC Conference on Computer and Communications Security (pp. 364-379).
\end{itemize}

\begin{itemize}
\item Bilge, L., & Dumitras, T. (2012). Before we knew it: an empirical study of zero-day attacks in the real world. In Proceedings of the 2012 ACM conference on Computer and communications security (pp. 833-844).
\end{itemize}

\begin{itemize}
\item Chandola, V., Banerjee, A., & Kumar, V. (2009). Anomaly detection: A survey. ACM computing surveys (CSUR), 41(3), 1-58.
\end{itemize}

\begin{itemize}
\item Veeramachaneni, K., Arnaldo, I., Korrapati, V., Bassias, C., & Li, K. (2016). AI^2: training a big data machine to defend. In 2016 IEEE 2nd International Conference on Big Data Security on Cloud (BigDataSecurity) (pp. 49-54). IEEE.
\end{itemize}

\end{document}